\section{Question 10.6}

\subsection{Question}
Find two examples of document filtering systems on the Web. How do they build a profile for your information need? Is the system static or adaptive?

\subsection{Answer}
My first example is Amazon (\url{www.amazon.com}).  This is an Internet marketplace where nearly any type of product can be purchased online and delivered to one's home.  The filtering the system performs is done as a recommendation of items it believes a user would be interested in purchasing.  It presents accurate recommendations by using historic data of items purchased together and calculating a correlation factor for pairs of items.  Using this correlation factor it can recommend items that are often purchased together when a user adds one of the pair to their online shopping cart.  For example, if one is shopping for mechanical pencils, it is likely that the user will also be interested in purchasing graphite refills or replacement erasers, because these items are often purchased together.  This system must be dynamic because it has to build correlations on the fly as item availability changes day to day and new correlations need to be created and updated to suit a dynamic market.

My second example is Netflix (\url{www.netflix.com}).  This is an online video streaming service that famously posted an open challenge for anyone that could provide a better recommendation system than their existing system.  The data used for the challenge was over 100 million ratings from 400 thousand users regarding 17 thousand movies.  Each data element was a quadruplet matching the form:\\

\texttt{<user, movie, date of grade, grade>}\\

The user and movie fields are simple unique identifiers, the date field could be any date type and the grade field is an integer from 1 to 5.  The goal of the recommender system is to predict the rating a user \textit{would} give to a movie they haven't yet watched and recommend movies the user hasn't yet seen that it believes the user would rate highly. 

It accomplishes accurate recommendations by using the movie grading records of its users.  First, it determines which users are similar to each other by computing a similarity measure between each pair of users based on the grades they have assigned to movies they have both watched.  If the pair of users have given similar grades for more movies their similarity measure will be higher.  The system then uses this measure to determine a ranking for each user of all other users ordered by their similarity measure.  Once this is done, the system can determine the most likely grade a user will assign to a movie they haven't seen based on grades similar users have assigned to that movie.  Users update their movie grades as they watch movies, so the data is constantly changing, which makes this a dynamic system as it needs to update movie preferences of users as new data becomes available to the system.