\section{Question 6.2}

\subsection{Question}
Create a simple spelling corrector based on the noisy channel model. Use a
single-word language model, and an error model where all errors with the same
edit distance have the same probability. Only consider edit distances of 1 or 2.
Implement your own edit distance calculator (example code can easily be found
on the Web)

\subsection{Approach}
The \texttt{spelling.py} script, found in Listing \ref{listing:spelling}, was created using the Python programming language \cite{python}.  Downloaded \texttt{big.txt} to calculate an example language model.  These words were counted and stored in a map that was compressed on disk using the pickle python library \cite{py:pickle}.\\

The process of determining a spelling correction is as follows:

\begin{enumerate}
    \item Count all the words contained in the example text file.  This count is used to determine the language model \(P(W)\) calculation.
    \item Take the input word and determine all existing (correctly spelled) words with edit distance one and two.
    \item With the assumption that shorter edit distances equate to a higher probability of being the correct spelling, select from the remaining set of (valid) words the one with the shortest edit distance and highest value for \(P(W)\).
\end{enumerate}

\(P(W)\) is calculated with the following formula:

\[P(W) = \frac{C_W}{N}\]

where \(C_W\) is the word count for word \(W\) and \(N\) is the sum of all word counts.

\subsection{Results}
Here is some sample output from the spelling.py script.

\lstinputlisting[language=Python, caption={spelling.py example output}, label=listing:spellingout]{code/spelling/out.txt}
