\section{Question 6.2}

\subsection{Question}
Create a simple spelling corrector based on the noisy channel model. Use a
single-word language model, and an error model where all errors with the same
edit distance have the same probability. Only consider edit distances of 1 or 2.
Implement your own edit distance calculator (example code can easily be found
on the Web)

\subsection{Approach}
Peter Norvig's noisy channel spelling correction algorithm \cite{pnorvig-spell} was used as the basis for this solution.  The \texttt{spelling.py} script, found in Listing \ref{listing:spelling}, was created as an implementation of this algorithm.  It was written with the Python programming language \cite{python}.

A large text file was downloaded from Mr. Norvig's website to calculate language model probability function \(P(W)\).  The words in the text file were counted and stored in a map that was compressed and saved on disk using the pickle python library \cite{py:pickle}.\\

\(P(W)\) is calculated with the following formula:

\[P(W) = \frac{C_W}{N}\]

where \(C_W\) is the word count for word \(W\) and \(N\) is the sum of all word counts.\\

The process of determining a spelling correction is as follows:

\begin{enumerate}
    \item Take the input word and determine all existing (correctly spelled) words with edit distance one and two.
    \item With the assumption that shorter edit distances equate to a higher probability of being the correct intended word, select from the set of words from the previous step the one with the shortest edit distance and highest value for \(P(W)\).
\end{enumerate}

\subsection{Results}
Here is some sample output from the spelling.py script.

\lstinputlisting[language=Python, caption={spelling.py example output}, label=listing:spellingout]{code/spelling/out.txt}
