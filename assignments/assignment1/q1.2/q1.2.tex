\section{Question 1.2}


\subsection{Question}
\textit{Site search} is another common application of search engines. In this case, search is restricted to the web pages at a given website. Compare site search to web search, vertical search, and enterprise search.


\subsection{Resources}
The textbook \textit{Search Engines: Information Retrieval in Practice} \cite{seirip} was used to answer this question.


\subsection{Answer}
Site search, as stated in the question, refers to a general search query, with results limited to pages within a particular host domain, or website.  Web search is defined as a general search over all the pages the web has to offer.  On one hand, site search is similar to web search because the query in unstructured.  On the other, site search is a far simpler beast than a general web search.  Being constrained to a particular host domain has a number of traits that make it an easier task to handle.  Firstly, the number of documents is orders of magnitude smaller for site search compared to web search, allowing the search results to be extremely fresh while not representing a daunting task of crawling the entire document collection.  The indexing process can also be completed much faster due to the drastically decreased number of documents.  Site search never has to worry about document \textit{spam}, items that should be excluded from any search as they are not intended to meet an information need, because the website maintaining the documents has complete control over which documents are to be indexed, leaving the decision made before the indexing process begins.

There are many examples of site-specific search features.  For example, the Internet Movie Database, an informational website in the domain of film and television, provides users with a site search feature to make finding a movie, TV show, or actor a simple task.  A vertical search engine may be about movies, but it is not necessarily going to have it's own repository of information within the domain of interest.

The key difference between a vertical search and a site-specific search is the search agent, or who is conducting the search.  With a vertical search the agent is retrieving results from many websites which are all outside the control of the vertical search engine itself.  For example, Google image search returns results that are all images, but they are not all contained on one of Google's own servers.  Thus, to build the search index on these documents, the vertical search provider must crawl the websites pertaining to their search topic, a task which requires care if the information is to be kept on the topic of interest.

This differs from a site-specific search in that this type search engine builds its index using documents controlled by the site administrators; the search itself is a service offered by a website to allow users more easy access to their own content.  This gives the proprietors of such a service an advantage in that web crawling is not required, eliminating many problems that come along with crawling the web -- non-uniformity of document models, possible network issues, storage concerns (the number of results are likely higher considering the vertical search engine must index on the entire web whereas the site specific search engine only indexes on the documents contained in that website).

Enterprise search is probably most similar to site search.  This is due to the same reasons site search is dissimilar to web search and vertical search -- the owner of the documents over which the search is made is also in control of the search engine itself.  The owner of the enterprise search engine is likely a company with a large repository of corporate documentation, thus the need for an enterprise search solution in the first place.  This allows the enterprise search provider to have the same benefits of site search described above: a smaller document collection (compared to web and vertical search) leads to faster indexing, query processing, and more fresh indexes.
