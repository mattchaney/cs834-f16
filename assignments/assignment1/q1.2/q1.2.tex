\section{Question 1.2}


\subsection{Question}
\textit{Site search} is another common application of search engines. In this case, search is restricted to the web pages at a given website. Compare site search to web search, vertical search, and enterprise search.


\subsection{Resources}
The textbook \textit{Search Engines: Information Retrieval in Practice} \cite{seirip} was used to answer this question.


\subsection{Answer}
Site search, as stated in the question, refers to a general search query, with results limited to pages within a particular host domain, or website.  This is similar to a vertical search when the site used as the site-constraint for a site-specific search is a website that focuses on a single topic, but 

Many websites provides a site-specific search feature, such as \hyperref[http://www.imdb.com]{Internet Movie Database}, an informational website in the domain of film and television.  This search functionality is provided to allow a user to quickly find a movie, TV show or actor they'd like to learn about.  A vertical search engine may be about movies, but it is not necessarily going to have it's own extensive repository of information within the domain of interest.\\

The key difference between a vertical search and a site-specific search is the search agent, or who is conducting the search, and who controls the document collection over which the search is made.  With a vertical search, the agent is retrieving results from many different websites, which are outside the control of the search agent.  Thus, to build the search index on these documents, the vertical search provider must crawl the websites pertaining to their search topic, a task which requires care if the information is to be kept on-topic.  This differs from a site-specific search in that site-specific searches are conducted by the site administrators; the search itself is a service offered by a website, using the document collection owned by the site administrators.  This gives the proprietors of such a service an advantage in that web crawling is not required, eliminating many problems public, general search engines face.  These include network bandwidth, not a problem if the index-builders run on the local machines.  

