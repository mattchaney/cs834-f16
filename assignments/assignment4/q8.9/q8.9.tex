\section{Question 8.9}

\subsection{Question}
For one query in the CACM collection, generate a ranking and calculate BPREF. Show that the two formulations of BPREF give the same value.

\subsection{Approach}
The first version of BPREF found in the text book is defined as follows:

\begin{equation}
\label{eq:1}
BPREF = \frac{1}{R} \sum_{d_r}(1 - \frac{N_{d_r}}{R})
\end{equation}

Where \(d_r\) is a relevant document, \(N_{d_r}\) gives the number of non-relevant documents that are ranked higher than document \(d_r\) and \(R\) is the size of the set of all relevant documents for the given query.\\

The second form is:

\begin{equation}
\label{eq:2}
BPREF = \frac{P}{P + Q}
\end{equation}


Where \(P\) is the number of preferences that agree and \(Q\) is the number of preferences that disagree.\\

The scripts \texttt{getrel.py} and \texttt{q89.py}, found in Listings \ref{listing:getrel} and \ref{listing:q89} were used to complete this exercise.

\subsubsection{Implementation}
The script \texttt{q89.py} was used as a driver program for this exercise and can be found in Listing \ref{listing:q89}.

\lstinputlisting[language=Python, caption={q89.py script}, label=listing:q89]{code/getrel/q89.py}

The first BPREF equation (\ref{eq:1}) was implemented as the \texttt{bpref1} function, which can be found in Listing \ref{listing:bpref1}.  This function iterates over the given set of retrieved documents and calculates 


\clearpage

\lstinputlisting[language=Python, caption={bpref1 function}, label=listing:bpref1, linerange={129-141}, firstnumber=129]{code/getrel/getrel.py}

The second BPREF equation (\ref{eq:1}) was implemented as the \texttt{bpref2} function, and can be found in Listing \ref{listing:bpref1}.

\lstinputlisting[language=Python, caption={bpref2 function}, label=listing:bpref2, linerange={143-154}, firstnumber=143]{code/getrel/getrel.py}


\subsection{Results}
Considering one example does not seem sufficient, a number of other queries were tested with the results shown in Table \ref{tab:q89out}

\begin{table}[H]
\centering
\begin{tabular}{ | l | l | l | }
\hline
Query & BPREF(1) & BPREF(2) \\
\hline
1 & 0.2 & 0.172413793103\\
\hline
5 & 0.078125 & 0.0704225352113\\
\hline
6 & 0.0 & 0.0\\
\hline
8 & 0.222222222222 & 0.181818181818\\
\hline
11 & 0.443213296399 & 0.354767184035\\
\hline
16 & 0.0692041522491 & 0.0626959247649\\
\hline
21 & 0.380165289256 & 0.304635761589\\
\hline
\end{tabular}
\caption{Table of BPREF values for a selection of CACM queries.}
\label{tab:bpref}
\end{table}
