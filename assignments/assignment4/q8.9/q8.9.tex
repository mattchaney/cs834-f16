\section{Question}
For one query in the CACM collection, generate a ranking and calculate BPREF. Show that the two formulations of BPREF give the same value.

\subsection{Approach}
The first version of BPREF found in the text book is defined as follows:

\begin{equation}
\nonumber
BPREF = \frac{1}{R} \sum_{d_r}(1 - \frac{N_{d_r}}{R})
\end{equation}

Where \(d_r\) is a relevant document, \(N_{d_r}\) gives the number of non-relevant documents that are ranked higher than document \(d_r\) and \(R\) is the size of the set of all relevant documents for the given query.\\

The second form is:

\begin{equation}
\nonumber
BPREF = \frac{P}{P + Q}
\end{equation}


Where \(P\) is the number of preferences that agree and \(Q\) is the number of preferences that disagree.\\

The scripts \texttt{getrel.py} and \texttt{q89.py}, found in Listings \ref{listing:getrel} and \ref{listing:q89} were used to complete this exercise.

\subsection{Results}
