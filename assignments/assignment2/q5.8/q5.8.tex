\section{Question 5.8}

\subsection{Question}
Write a program that can build a simple inverted index of a set of text documents. Each inverted list will contain the file names of the documents that contain that word.

Suppose the file A contains the text ``the quick brown fox'', and file B contains ``the slow blue fox''. 

\begin{verbatim}
The output of your program would be:
% ./your-program A B
blue B
brown A
fox A B
quick A
slow B
the A B
\end{verbatim}

\subsection{Resources}
The textbook \textit{Search Engines: Information Retrieval in Practice} \cite{seirip}, the Python programming language \cite{python} with the python libraries Beautiful Soup \cite{py:beautifulsoup} and NLTK \cite{py:nltk} were used to answer this question.

\subsection{Answer}
Again, the filevisitor.py script found in Listing \ref{listing:filevisitor} was modified to create the inverted index while visiting each file from the small Wikipedia collection.  Afterwards, the search.py \ref{listing:search} script can be used to search for terms within the document collection.

Example output for searching the inverted index for the terms ``Guy'' and ``Gal'' can be found in Listing \ref{listing:searchresults}.\\

\lstinputlisting[language=bash, caption={Search Results for Terms ``Guy'' and ``Gal''}, label=listing:searchresults]{q5.8/search.txt}
